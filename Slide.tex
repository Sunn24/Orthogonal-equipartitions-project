\documentclass[11pt,aspectratio=43,xcolor={dvipsnames},notheorems]{beamer}
\usepackage{emoji} % LuaLaTeX
%%%%%%%Begin thai template%%%%%%%
% \usepackage[no-math]{fontspec}
% \usepackage{xunicode}
% \usepackage{xltxtra}
% \XeTeXlinebreaklocale "th"
% \XeTeXlinebreakskip = 0pt plus 0.5pt
% \defaultfontfeatures{Mapping=tex-text} 
% \newfontfamily{\thaifont}[Path = ./Header/Fonts/,BoldFont={THSarabunNew Bold.ttf},ItalicFont={THSarabunNew Italic.ttf},BoldItalicFont={THSarabunNew BoldItalic.ttf},Scale=MatchLowercase]{THSarabunNew.ttf}
% \newfontfamily{\englishfont}[Path = ./Header/Fonts/,BoldFont={cmunbx.ttf},ItalicFont={cmunti.ttf},BoldItalicFont={cmunbi.ttf},Scale=MatchLowercase]{cmunrm.ttf}
% \usepackage[Latin,Thai]{ucharclasses}
% \setTransitionTo{Thai}{\thaifont}
% \setTransitionFrom{Thai}{\englishfont}
%%%%%%%end thai template%%%%%%%
\usepackage{setspace}
\setstretch{1.1}
% \setlength{\parindent}{2em}
%\setlength{\emergencystretch}{2em}
%สำหรับการใช้ฟ้อนThai-Eng%

\usetheme{PaloAlto}
%\usecolortheme{orchid}
\usefonttheme{serif} 

\definecolor{msl}{HTML} {ff77bc}
% \definecolor{msl}{RGB}{33,144,189}
\definecolor{blueblack1}{HTML}{003366}
\definecolor{blueblack2}{RGB}{0,0,80}
% \usecolortheme[RGB={179,157,219}]{structure}
\usecolortheme[named=msl]{structure}

% \setbeamersize{sidebar width left=2cm}
\setbeamercolor{frametitle}{bg=msl!20!msl}
\setbeamercolor{sidebar}{bg=msl} % sidebar text color
\setbeamercolor{title in sidebar}{fg=white}

\colorlet{titleright}{msl!30!white} % sidebar and headline color
% \colorlet{titleright}{msl!30!black}
\colorlet{titlemid}{msl}
\colorlet{titleleft}{msl}

% \makeatletter
% \setlength\beamer@sidebarwidth{1.5cm}
% \makeatother

\setbeamertemplate{title page}[default][colsep=-4bp,rounded=true]
\setbeamertemplate{blocks}[rounded][shadow=false]

\makeatletter
% shading on background %%
\setbeamertemplate{background canvas}[vertical shading][bottom=msl!10,top=msl!2]
% \setbeamertemplate{background canvas}[vertical shading][top=msl!30,bottom=white!10]

\pgfdeclarehorizontalshading[titleleft,titleright]{beamer@frametitleshade}{\paperheight}{%
    color(\beamer@sidebarwidth)=(titlemid);%
    color(\paperwidth)=(titleright)%
}

\pgfdeclareverticalshading[titleleft,titleright]{beamer@sidebar}{\beamer@sidebarwidth}{%
    color(0pt)=(titleright);%
    color(\sidebarheight)=(titleleft)%
}

%% shading on headline %%
\setbeamertemplate{headline}{%
    \begin{beamercolorbox}[wd=\paperwidth]{frametitle}%
        \ifx\beamer@sidebarside\beamer@lefttext%
        \else%
        \hfill%
        \fi%
        \ifdim\beamer@sidebarwidth>0pt%  
        \usebeamercolor[bg]{logo}%
        \begin{pgfpicture}{0pt}{0pt}{\paperwidth}{\beamer@headheight}%
            \usebeamercolor{frametitle right}%
            \pgfpathrectangle{\pgfpointorigin}{\pgfpoint{\paperwidth}{\beamer@headheight}}%
            \pgfusepath{clip}%
            \pgftext[left,base]{\pgfuseshading{beamer@frametitleshade}}%
        \end{pgfpicture}%
        \vskip-\beamer@headheight%
        \vrule width\beamer@sidebarwidth height \beamer@headheight%
        \hskip-\beamer@sidebarwidth%
        \hbox to \beamer@sidebarwidth{\hss\vbox to
            \beamer@headheight{\vss\hbox{\color{fg}\insertlogo}\vss}\hss}%
        \else%
        \vrule width0pt height \beamer@headheight%  
        \fi%
    \end{beamercolorbox}
}
%% shading on sidebar %%
\setbeamertemplate{sidebar canvas left}{
    \pgfuseshading{beamer@sidebar}
}

\makeatother
\usepackage{tikz}
\usepackage{pgfplots}
\pgfplotsset{compat=newest}
\usetikzlibrary{arrows,angles,quotes}
\usepackage{caption}
\usepackage{subcaption}
\usepackage{graphicx}
\graphicspath{{.}}
\usepackage{multicol}
% \setbeamercovered{transparent=10}

% \usebackgroundtemplate{%
% \tikz[overlay,remember picture] \node[opacity=0.2, at=(current page.center)] {
%    \includegraphics[height=\paperheight,width=\paperwidth]{Mahidollogo.png}};
% }


% \addtobeamertemplate{navigation symbols}{}{%
%     \usebeamerfont{footline}%
%     \usebeamercolor[fg]{footline}%
%     \hspace{1em}%
%     \insertframenumber/\inserttotalframenumber
% }
\addtobeamertemplate{navigation symbols}{}{%
  \usebeamerfont{footline}%
  \usebeamercolor[fg]{footline}%
  \hspace{1em}%
  {\fontsize{12}{13} \selectfont\insertframenumber/\inserttotalframenumber}
}

%\logo{\includegraphics[height=1cm]{logo2.png}}
\usepackage{hyperref}
\hypersetup{
    colorlinks=true,
    linkcolor=black,
    filecolor=magenta,      
    urlcolor=blue
    }

% \usepackage{bibentry}
\usepackage{booktabs} % Allows the use of \toprule, 


\title[Orthogonal Equipartitions]{Orthogonal Equipartitions of 3-color Points in the Plane}
% \subtitle{for SCMA 490 Seminar}
% \author[Krittapas N.]{Krittapas Ngammuengman 6305146}
\author{Krittapas N.}
\date{May 17, 2024}

% math mode package %
\usepackage{amsfonts}
\usepackage{amsmath}
\usepackage{mathtools}
\usepackage{amssymb}

% theoremstyle %
\usepackage{amsthm}
\theoremstyle{definition}
\newtheorem{definition}{บทนิยาม}[]

\theoremstyle{plain}
\newtheorem{theorem}{ทฤษฎีบท}[]
\newtheorem{lemma}[theorem]{บทตั้ง}
\newtheorem{proposition}[theorem]{ทฤษฎีบทประกอบ}
\newtheorem{corollary}[theorem]{บทแทรก}
\newtheorem{observation}{ข้อสังเกต}[]

\theoremstyle{remark}
\newtheorem{example}{ตัวอย่าง}
\newtheorem{exercise}{แบบฝึกหัด}
\newtheorem*{remark}{หมายเหตุ}
% \newtheorem{note}{บันทึก}
\setbeamertemplate{theorems}[numbered]
% \setbeamertemplate{theorems}[ams style] 

\newcommand{\R}{\mathbb{R}}
\newcommand{\N}{\mathbb{N}}
\DeclareMathOperator{\conv}{conv}

\usepackage{animate}

\begin{document}
\nocite{*}
\maketitle
% \begin{frame}{Outline}
% \tableofcontents
% \end{frame}

\section{Introduction}
\begin{frame}[allowframebreaks]{Introduction}
% \begin{enumerate}
%     \item 2-Color Points in the plane
%     \item Thm 2 (Convex hull)
%     \item Thm 3 (Semi-Rectangular cut)
%     \item One(part) of Thm 4 (Restrict the number of segment)
%     \item main problem
% \end{enumerate}
% \end{frame}
% \begin{frame}
\begin{figure}[ht]
    \centering
\begin{tikzpicture}[thick, scale=0.52]
%%%%% Axis %%%%%
% \draw[help lines, color=gray!30, dashed] (-4.9,-4.9) grid (4.9,4.9);
\draw[->,ultra thick] (-6,-5)--(5,-5) ;
\draw[->,ultra thick] (-5,-6)--(-5,5) ;
%%%%% Point %%%%%
\draw[fill=red,draw=black] (-3.5,-3.5) circle (3pt);
\draw[fill=red,draw=black] (-1,3) circle (3pt);
\draw[fill=red,draw=black] (1,-3) circle (3pt);
\draw[fill=red,draw=black] (-3,2) circle (3pt);
\draw[fill=red,draw=black] (2,0) circle (3pt);
\draw[fill=red,draw=black] (-0.5,0.5) circle (3pt);
\draw[fill=red,draw=black] (1,4) circle (3pt);
\draw[fill=red,draw=black] (-4,-1) circle (3pt);

\draw[fill=blue,draw=black] (3,-2) circle (3pt);
\draw[fill=blue,draw=black] (2.5,2) circle (3pt);
\draw[fill=blue,draw=black] (-2,-4) circle (3pt);
\draw[fill=blue,draw=black] (3,2.5) circle (3pt);
\draw[fill=blue,draw=black] (-2.5,-3) circle (3pt);
\draw[fill=blue,draw=black] (-1.5,2) circle (3pt);
\draw[fill=blue,draw=black] (2.5,-3.5) circle (3pt);
\draw[fill=blue,draw=black] (-2.5,3) circle (3pt);
\draw[fill=blue,draw=black] (0.5,4.5) circle (3pt);
\draw[fill=blue,draw=black] (-0.5,1) circle (3pt);
\draw[fill=blue,draw=black] (-4.5,-1.5) circle (3pt);
\draw[fill=blue,draw=black] (4,-4) circle (3pt);
\draw[fill=blue,draw=black] (0.5,0.5) circle (3pt);
\draw[fill=blue,draw=black] (-1,4) circle (3pt);
\draw[fill=blue,draw=black] (-3,-2) circle (3pt);
\draw[fill=blue,draw=black] (4,-1) circle (3pt);
\end{tikzpicture}
\caption{2-color points in the plane in general position.}
\end{figure}
% \begin{theorem}[]
%     Let $R$ and $B$ be two disjoint sets of $an$ red points and $bn$ blue points in the plane respectively, where $a\ge 1$ and $b\ge 1$. then $R\cup B$ can be partitioned into disjoint subsets $P_1\cup P_2\cup\cdots \cup P_n$ satisfying the following conditions:
%     \begin{enumerate}
%         \item $\conv(P_i)\cap\conv(P_j)=\varnothing$ for all $1\le i<j\le q$,
%         \item every $P_i$ contains exactly $a$ red points and $b$ blue points.
%     \end{enumerate}
% \end{theorem}
% \end{frame}
% \begin{frame}
\begin{figure}[ht]
    \centering
\begin{tikzpicture}[thick, scale=0.52]
%%%%% Axis %%%%%
% \draw[help lines, color=gray!30, dashed] (-4.9,-4.9) grid (4.9,4.9);
\draw[->,ultra thick] (-6,-5)--(5,-5) ;
\draw[->,ultra thick] (-5,-6)--(-5,5) ;
\draw[densely dashed,color=red,line width=0.75pt,opacity=0.8] (-2,-4)--(-3,-2)--(-4,-1)--(-4.5,-1.5)--(-3.5,-3.5)--cycle;
\draw[densely dashed,color=red,line width=0.75pt,opacity=0.8] (4,-4)--(4,-1)--(2,0)--(1,-3)--cycle;
\draw[densely dashed,color=red,line width=0.75pt,opacity=0.8] (0.5,0.5)--(0.5,4.5)--(-1,4)--(-2.5,3)--(-3,2)--(-0.5,0.5)--cycle;
\draw[densely dashed,color=red,line width=0.75pt,opacity=0.8] (3,2.5)--(1,4)--(2.5,2)--cycle;
%%%%% Point %%%%%
\draw[fill=red,draw=black] (-3.5,-3.5) circle (3pt);
\draw[fill=red,draw=black] (-1,3) circle (3pt);
\draw[fill=red,draw=black] (1,-3) circle (3pt);
\draw[fill=red,draw=black] (-3,2) circle (3pt);
\draw[fill=red,draw=black] (2,0) circle (3pt);
\draw[fill=red,draw=black] (-0.5,0.5) circle (3pt);
\draw[fill=red,draw=black] (1,4) circle (3pt);
\draw[fill=red,draw=black] (-4,-1) circle (3pt);

\draw[fill=blue,draw=black] (3,-2) circle (3pt);
\draw[fill=blue,draw=black] (2.5,2) circle (3pt);
\draw[fill=blue,draw=black] (-2,-4) circle (3pt);
\draw[fill=blue,draw=black] (3,2.5) circle (3pt);
\draw[fill=blue,draw=black] (-2.5,-3) circle (3pt);
\draw[fill=blue,draw=black] (-1.5,2) circle (3pt);
\draw[fill=blue,draw=black] (2.5,-3.5) circle (3pt);
\draw[fill=blue,draw=black] (-2.5,3) circle (3pt);
\draw[fill=blue,draw=black] (0.5,4.5) circle (3pt);
\draw[fill=blue,draw=black] (-0.5,1) circle (3pt);
\draw[fill=blue,draw=black] (-4.5,-1.5) circle (3pt);
\draw[fill=blue,draw=black] (4,-4) circle (3pt);
\draw[fill=blue,draw=black] (0.5,0.5) circle (3pt);
\draw[fill=blue,draw=black] (-1,4) circle (3pt);
\draw[fill=blue,draw=black] (-3,-2) circle (3pt);
\draw[fill=blue,draw=black] (4,-1) circle (3pt);
\end{tikzpicture}
\caption{Balanced
partition of convex sets.}
\end{figure}

\begin{figure}[ht]
    \centering
\begin{tikzpicture}[thick, scale=0.52]
%%%%% Axis %%%%%
\draw[help lines, color=gray!60, dashed] (-4.9,-4.9) grid (4.9,4.9);
\draw[->,ultra thick] (-6,-5)--(5,-5) ;
\draw[->,ultra thick] (-5,-6)--(-5,5) ;
\draw[densely dashed,color=red,line width=0.75pt,opacity=0.8] (-0.75,5)--(-0.75,3.5)--(-1.25,3.5)--(-1.25,-2.75)--(1.25,-2.75)--(1.25,-3.25)--(5,-3.25);
%%%%% Point %%%%%
\draw[fill=white,draw=red] (-1.25,-2.75) circle (3pt);

\draw[fill=red,draw=black] (-3.5,-3.5) circle (3pt);
\draw[fill=red,draw=black] (-1,3) circle (3pt);
\draw[fill=red,draw=black] (1,-3) circle (3pt);
\draw[fill=red,draw=black] (-3,2) circle (3pt);
\draw[fill=red,draw=black] (2,0) circle (3pt);
\draw[fill=red,draw=black] (-0.5,0.5) circle (3pt);
\draw[fill=red,draw=black] (1,4) circle (3pt);
\draw[fill=red,draw=black] (-4,-1) circle (3pt);

\draw[fill=blue,draw=black] (3,-2) circle (3pt);
\draw[fill=blue,draw=black] (2.5,2) circle (3pt);
\draw[fill=blue,draw=black] (-2,-4) circle (3pt);
\draw[fill=blue,draw=black] (3,2.5) circle (3pt);
\draw[fill=blue,draw=black] (-2.5,-3) circle (3pt);
\draw[fill=blue,draw=black] (-1.5,2) circle (3pt);
\draw[fill=blue,draw=black] (2.5,-3.5) circle (3pt);
\draw[fill=blue,draw=black] (-2.5,3) circle (3pt);
\draw[fill=blue,draw=black] (0.5,4.5) circle (3pt);
\draw[fill=blue,draw=black] (-0.5,1) circle (3pt);
\draw[fill=blue,draw=black] (-4.5,-1.5) circle (3pt);
\draw[fill=blue,draw=black] (4,-4) circle (3pt);
\draw[fill=blue,draw=black] (0.5,0.5) circle (3pt);
\draw[fill=blue,draw=black] (-1,4) circle (3pt);
\draw[fill=blue,draw=black] (-3,-2) circle (3pt);
\draw[fill=blue,draw=black] (4,-1) circle (3pt);
\end{tikzpicture}
\caption{Balanced partition of semi-rectangular
cut.}
\end{figure}

\begin{figure}[ht]
    \centering
\begin{tikzpicture}[thick, scale=0.52]
%%%%% Axis %%%%%
\draw[help lines, color=gray!30, dashed] (-4.9,-4.9) grid (4.9,4.9);
\draw[->,ultra thick] (-6,-5)--(5,-5) ;
\draw[->,ultra thick] (-5,-6)--(-5,5) ;
\draw[densely dashed,color=red,line width=0.75pt,opacity=0.8] (-1.75,5)--(-1.75,-1.25)--(5,-1.25);
%%%%% Point %%%%%
\draw[fill=white,draw=red] (-1.75,-1.25) circle (3pt);

\draw[fill=red,draw=black] (-3.5,-3.5) circle (3pt);
\draw[fill=red,draw=black] (-1,3) circle (3pt);
\draw[fill=red,draw=black] (1,-3) circle (3pt);
\draw[fill=red,draw=black] (-3,2) circle (3pt);
\draw[fill=red,draw=black] (2,0) circle (3pt);
\draw[fill=red,draw=black] (-0.5,0.5) circle (3pt);
\draw[fill=red,draw=black] (1,4) circle (3pt);
\draw[fill=red,draw=black] (-4,-1) circle (3pt);

\draw[fill=blue,draw=black] (3,-2) circle (3pt);
\draw[fill=blue,draw=black] (2.5,2) circle (3pt);
\draw[fill=blue,draw=black] (-2,-4) circle (3pt);
\draw[fill=blue,draw=black] (3,2.5) circle (3pt);
\draw[fill=blue,draw=black] (-2.5,-3) circle (3pt);
\draw[fill=blue,draw=black] (-1.5,2) circle (3pt);
\draw[fill=blue,draw=black] (2.5,-3.5) circle (3pt);
\draw[fill=blue,draw=black] (-2.5,3) circle (3pt);
\draw[fill=blue,draw=black] (0.5,4.5) circle (3pt);
\draw[fill=blue,draw=black] (-0.5,1) circle (3pt);
\draw[fill=blue,draw=black] (-4.5,-1.5) circle (3pt);
\draw[fill=blue,draw=black] (4,-4) circle (3pt);
\draw[fill=blue,draw=black] (0.5,0.5) circle (3pt);
\draw[fill=blue,draw=black] (-1,4) circle (3pt);
\draw[fill=blue,draw=black] (-3,-2) circle (3pt);
\draw[fill=blue,draw=black] (4,-1) circle (3pt);
\end{tikzpicture}
\caption{Balanced partition of 1 vertical and 1 horizontal segments.}
\end{figure}
\end{frame}
\begin{frame}{Main Problem}
\begin{block}{Problem 1}
    Let $k>2$ be an integer. Given a set of $k-$color points in general position in the plane. What is a smallest number\\
    of line segments for equitable partitioning $k-$color into\\
    two regions?
    % $$\left\lfloor\dfrac{n_i}{2}\right\rfloor \text{ or } \left\lceil\dfrac{n_i}{2}\right\rceil$$
    % for each color $i=1,\,2,\,\ldots,\,k$.

    
    % Let $k\ge 3$ and $n\ge 1$ integers. For any set $S$ of $n$ points in general position in the plane. 
\end{block}
\end{frame}
\section{Methodology}
\begin{frame}[allowframebreaks]{Methodology}
% \begin{enumerate}
%     \item รูปที่ต้องการ 
%     % \item random
%     \item first vertical
%     \item ฺfor 3 segment
%     \item for 5 segment
% \end{enumerate}
\begin{figure}[ht]
    \centering
\begin{tikzpicture}[thick, scale=0.52]
%%%%% Axis %%%%%
\draw[help lines, color=gray!30, dashed] (-4.9,-4.9) grid (4.9,4.9);
\draw[->,ultra thick] (-6,-5)--(5,-5) ;
\draw[->,ultra thick] (-5,-6)--(-5,5) ;
%%%%% Point %%%%%
\draw[fill=lime,draw=black] (2.5,1) circle (3pt);
\draw[fill=lime,draw=black] (-2,-4) circle (3pt);
\draw[fill=lime,draw=black] (3,2.5) circle (3pt);
\draw[fill=lime,draw=black] (-2.5,-3) circle (3pt);
\draw[fill=lime,draw=black] (-1.5,2) circle (3pt);
\draw[fill=lime,draw=black] (-1.5,-3.5) circle (3pt);

\draw[fill=red,draw=black] (-1,3) circle (3pt);
\draw[fill=red,draw=black] (1,-3) circle (3pt);
\draw[fill=red,draw=black] (-3,2) circle (3pt);
\draw[fill=red,draw=black] (2,0) circle (3pt);
\draw[fill=red,draw=black] (-0.5,0.5) circle (3pt);
\draw[fill=red,draw=black] (1,4) circle (3pt);
\draw[fill=red,draw=black] (-4,-1) circle (3pt);
\draw[fill=red,draw=black] (3,-2) circle (3pt);

\draw[fill=blue,draw=black] (0.5,4.5) circle (3pt);
\draw[fill=blue,draw=black] (-0.5,1) circle (3pt);
\draw[fill=blue,draw=black] (-4.5,-1.5) circle (3pt);
\draw[fill=blue,draw=black] (4,-4) circle (3pt);
\draw[fill=blue,draw=black] (0.5,0.5) circle (3pt);
\draw[fill=blue,draw=black] (-1,4) circle (3pt);
\draw[fill=blue,draw=black] (-3,-2) circle (3pt);
\draw[fill=blue,draw=black] (4,-1) circle (3pt);
\end{tikzpicture}
\caption{The 3-color points in general position in the plane.}
\label{ex1}
\end{figure}
% \newpage
% \textbf{}\vfill
% \begin{block}{}
%     abc
% \end{block}
% \vfill\textbf{}
% \newpage
\begin{figure}[!ht]
    \centering
\begin{tikzpicture}[thick, scale=0.52]
%%%%% Axis %%%%%
\draw[help lines, color=gray!30, dashed] (-4.9,-4.9) grid (4.9,4.9);
\draw[->,ultra thick] (-6,-5)--(5,-5) ;
\draw[->,ultra thick] (-5,-6)--(-5,5) ;
%%%%% Point %%%%%
\draw[fill=lime,draw=black] (2.5,1) circle (3pt);
\draw[fill=lime,draw=black] (-2,-4) circle (3pt);
\draw[fill=lime,draw=black] (3,2.5) circle (3pt);
\draw[fill=lime,draw=black] (-2.5,-3) circle (3pt);
\draw[fill=lime,draw=black] (-1.5,2) circle (3pt);
\draw[fill=lime,draw=black] (-1.5,-3.5) circle (3pt);

\draw[fill=red,draw=black] (-1,3) circle (3pt);
\draw[fill=red,draw=black] (1,-3) circle (3pt);
\draw[fill=red,draw=black] (-3,2) circle (3pt);
\draw[fill=red,draw=black] (2,0) circle (3pt);
\draw[fill=red,draw=black] (-0.5,0.5) circle (3pt);
\draw[fill=red,draw=black] (1,4) circle (3pt);
\draw[fill=red,draw=black] (-4,-1) circle (3pt);
\draw[fill=red,draw=black] (3,-2) circle (3pt);

\draw[fill=blue,draw=black] (0.5,4.5) circle (3pt);
\draw[fill=blue,draw=black] (-0.5,1) circle (3pt);
\draw[fill=blue,draw=black] (-4.5,-1.5) circle (3pt);
\draw[fill=blue,draw=black] (4,-4) circle (3pt);
\draw[fill=blue,draw=black] (0.5,0.5) circle (3pt);
\draw[fill=blue,draw=black] (-1,4) circle (3pt);
\draw[fill=blue,draw=black] (-3,-2) circle (3pt);
\draw[fill=blue,draw=black] (4,-1) circle (3pt);
\draw[densely dashed,color=blue,line width=1pt,opacity=1] (-5,-2)--(2,-2);
\draw[densely dashed,color=red!90!black,line width=0.75pt,opacity=1] (2,-5)--(2,-2)--(5,-2);
\node at (-5.5,-2) {$i$};
\node at (2,-5.5) {$j$};
\node at (-1.5,-1.5) {$a_{ij}$};
\end{tikzpicture}
\caption{$a_{ij}$ is the number of each color points that lie on $y=i$ from $x=0$ to $x=j$}
\label{ex2}
\end{figure}
\begin{figure}[!h]
    \centering
\begin{tikzpicture}[thick, scale=0.52]
%%%%% Axis %%%%%
\draw[help lines, color=gray!30, dashed] (-4.9,-4.9) grid (4.9,4.9);
\draw[->,ultra thick] (-6,-5)--(5,-5) ;
\draw[->,ultra thick] (-5,-6)--(-5,5) ;
%%%%% Blue Point %%%%%
\draw[fill=lime,draw=black] (2.5,1) circle (3pt);
\draw[fill=lime,draw=black] (-2,-4) circle (3pt);
\draw[fill=lime,draw=black] (3,2.5) circle (3pt);
\draw[fill=lime,draw=black] (-2.5,-3) circle (3pt);
\draw[fill=lime,draw=black] (-1.5,2) circle (3pt);
\draw[fill=lime,draw=black] (-1.5,-3.5) circle (3pt);

\draw[fill=red,draw=black] (-1,3) circle (3pt);
\draw[fill=red,draw=black] (1,-3) circle (3pt);
\draw[fill=red,draw=black] (-3,2) circle (3pt);
\draw[fill=red,draw=black] (2,0) circle (3pt);
\draw[fill=red,draw=black] (-0.5,0.5) circle (3pt);
\draw[fill=red,draw=black] (1,4) circle (3pt);
\draw[fill=red,draw=black] (-4,-1) circle (3pt);
\draw[fill=red,draw=black] (3,-2) circle (3pt);

\draw[fill=blue,draw=black] (0.5,4.5) circle (3pt);
\draw[fill=blue,draw=black] (-0.5,1) circle (3pt);
\draw[fill=blue,draw=black] (-4.5,-1.5) circle (3pt);
\draw[fill=blue,draw=black] (4,-4) circle (3pt);
\draw[fill=blue,draw=black] (0.5,0.5) circle (3pt);
\draw[fill=blue,draw=black] (-1,4) circle (3pt);
\draw[fill=blue,draw=black] (-3,-2) circle (3pt);
\draw[fill=blue,draw=black] (4,-1) circle (3pt);
\foreach \i in {-5,-4.5,...,4.5,5}{
\draw[densely dashed,color=blue,line width=0.5pt,opacity=0.6] (\i,-5)--(\i,5);
}
\node at (0,-5.75) {$S_j=\sum_{i=1}^na_{ij}$};
\end{tikzpicture}
\caption{$S_j$ is a sum of number of each color points along left side of a line $x=j$.}
\label{ex3}
\end{figure}
\vspace{-2\baselineskip}
\begin{figure}[!ht]
    \centering
\begin{tikzpicture}[thick, scale=0.52]
%%%%% Axis %%%%%
\draw[help lines, color=gray!30, dashed] (-4.9,-4.9) grid (4.9,4.9);
\draw[->,ultra thick] (-6,-5)--(5,-5) ;
\draw[->,ultra thick] (-5,-6)--(-5,5) ;
\draw[densely dashed,color=red,line width=1pt,opacity=1] (-0.5,-5)--(-0.5,5);
%%%%% Point %%%%%
\draw[fill=lime,draw=black] (2.5,1) circle (3pt);
\draw[fill=lime,draw=black] (-2,-4) circle (3pt);
\draw[fill=lime,draw=black] (3,2.5) circle (3pt);
\draw[fill=lime,draw=black] (-2.5,-3) circle (3pt);
\draw[fill=lime,draw=black] (-1.5,2) circle (3pt);
\draw[fill=lime,draw=black] (-1.5,-3.5) circle (3pt);

\draw[fill=red,draw=black] (-1,3) circle (3pt);
\draw[fill=red,draw=black] (1,-3) circle (3pt);
\draw[fill=red,draw=black] (-3,2) circle (3pt);
\draw[fill=red,draw=black] (2,0) circle (3pt);
\draw[fill=red,draw=black] (-0.5,0.5) circle (3pt);
\draw[fill=red,draw=black] (1,4) circle (3pt);
\draw[fill=red,draw=black] (-4,-1) circle (3pt);
\draw[fill=red,draw=black] (3,-2) circle (3pt);

\draw[fill=blue,draw=black] (0.5,4.5) circle (3pt);
\draw[fill=blue,draw=black] (-0.5,1) circle (3pt);
\draw[fill=blue,draw=black] (-4.5,-1.5) circle (3pt);
\draw[fill=blue,draw=black] (4,-4) circle (3pt);
\draw[fill=blue,draw=black] (0.5,0.5) circle (3pt);
\draw[fill=blue,draw=black] (-1,4) circle (3pt);
\draw[fill=blue,draw=black] (-3,-2) circle (3pt);
\draw[fill=blue,draw=black] (4,-1) circle (3pt);
\foreach \i in {-5,-4.5,...,4.5,5}{
\draw[densely dashed,color=blue,line width=0.5pt,opacity=0.3] (\i,-5)--(\i,5);
}
\end{tikzpicture}
\caption{A first vertical line $x=j$.}
\end{figure}
\begin{figure}[!ht]
    \centering
\begin{tikzpicture}[thick, scale=0.52]
%%%%% Axis %%%%%
\draw[help lines, color=gray!30, dashed] (-4.9,-4.9) grid (4.9,4.9);
\draw[->,ultra thick] (-6,-5)--(5,-5) ;
\draw[->,ultra thick] (-5,-6)--(-5,5) ;
\draw[densely dashed,color=red,line width=1pt,opacity=1] (-0.25,-5)--(-0.25,5);
%%%%% Point %%%%%
\draw[fill=lime,draw=black] (2.5,1) circle (3pt);
\draw[fill=lime,draw=black] (-2,-4) circle (3pt);
\draw[fill=lime,draw=black] (3,2.5) circle (3pt);
\draw[fill=lime,draw=black] (-2.5,-3) circle (3pt);
\draw[fill=lime,draw=black] (-1.5,2) circle (3pt);
\draw[fill=lime,draw=black] (-1.5,-3.5) circle (3pt);

\draw[fill=red,draw=black] (-1,3) circle (3pt);
\draw[fill=red,draw=black] (1,-3) circle (3pt);
\draw[fill=red,draw=black] (-3,2) circle (3pt);
\draw[fill=red,draw=black] (2,0) circle (3pt);
\draw[fill=red,draw=black] (-0.5,0.5) circle (3pt);
\draw[fill=red,draw=black] (1,4) circle (3pt);
\draw[fill=red,draw=black] (-4,-1) circle (3pt);
\draw[fill=red,draw=black] (3,-2) circle (3pt);

\draw[fill=blue,draw=black] (0.5,4.5) circle (3pt);
\draw[fill=blue,draw=black] (-0.5,1) circle (3pt);
\draw[fill=blue,draw=black] (-4.5,-1.5) circle (3pt);
\draw[fill=blue,draw=black] (4,-4) circle (3pt);
\draw[fill=blue,draw=black] (0.5,0.5) circle (3pt);
\draw[fill=blue,draw=black] (-1,4) circle (3pt);
\draw[fill=blue,draw=black] (-3,-2) circle (3pt);
\draw[fill=blue,draw=black] (4,-1) circle (3pt);
% \foreach \i in {-5,-4.5,...,4.5,5}{
% \draw[densely dashed,color=blue,line width=0.5pt,opacity=0.3] (\i,-5)--(\i,5);
% }
\end{tikzpicture}
\caption{A first vertical line $x=j$.}
\end{figure}
\begin{block}{Verify a solution}
    A first vertical line $x=j$ is a solution when $$\left|\dfrac{S_n}{2}-S_j\right|\le \dfrac{1}{2},$$ 
    for every color of points.
\end{block}
\end{frame}
\subsection{3 Line Segments}
\begin{frame}[allowframebreaks]{3 Line Segments}
\begin{figure}[!ht]
\centering
\begin{subfigure}{0.49\textwidth}
\centering
\begin{tikzpicture}[thick, scale=0.4]
%%%%% Axis %%%%%
\draw[help lines, color=gray!30, dashed] (-4.9,-4.9) grid (4.9,4.9);
\draw[->,ultra thick] (-6,-5)--(5,-5) ;
\draw[->,ultra thick] (-5,-6)--(-5,5) ;
\draw[densely dashed,color=red,line width=1pt,opacity=1](-2,-5)--(-2,0)--(2,0)--(2,5);
\foreach \i in {-5,-4.5,...,-3,-2.5}{
\draw[densely dashed,color=blue,line width=0.5pt,opacity=1] (\i,-5)--(\i,5);
}
\foreach \i in {-2,-1.5,...,1,1.5}{
\draw[densely dashed,color=blue,line width=0.5pt,opacity=1] (\i,0)--(\i,5);
}
\node at (-5.5,0) {$\color{red!80!black}i_1$};
\node at (-2,-5.5) {$\color{red!80!black}j_1$};
\node at (2,-5.5) {$j$};
\end{tikzpicture}
\end{subfigure}
\begin{subfigure}{0.49\textwidth}
\centering
\begin{tikzpicture}[thick, scale=0.4]
%%%%% Axis %%%%%
\draw[help lines, color=gray!30, dashed] (-4.9,-4.9) grid (4.9,4.9);
\draw[->,ultra thick] (-6,-5)--(5,-5) ;
\draw[->,ultra thick] (-5,-6)--(-5,5) ;
\draw[densely dashed,color=red,line width=1pt,opacity=1](-2,5)--(-2,0)--(2,0)--(2,-5);
\foreach \i in {-5,-4.5,...,-3,-2.5}{
\draw[densely dashed,color=blue,line width=0.5pt,opacity=1] (\i,-5)--(\i,5);
}
\foreach \i in {-2,-1.5,...,1,1.5}{
\draw[densely dashed,color=blue,line width=0.5pt,opacity=1] (\i,-5)--(\i,0);
}
\node at (-5.5,0) {$\color{red!80!black}i_1$};
\node at (-2,-5.5) {$\color{red!80!black}j_1$};
\node at (2,-5.5) {$j$};

\end{tikzpicture}
\end{subfigure}
\caption{2 vertical and 1 horizontal segments.}
\end{figure}
\begin{block}{Convert into equation}
\begin{equation}
    V=\sum_{i=0}^{i_1}a_{ij}+\sum_{i=i_1}^{n}a_{ij_1}\quad\text{or}\quad V=\sum_{i=0}^{i_1}a_{ij_1} +\sum_{i=i_1}^{n}a_{ij},
\end{equation}
where $0\le j_1\le j$ for each color of points.
\end{block}
\begin{block}{Verify a solution}
    This partition is a solution when $$\left|\dfrac{S_n}{2}-V\right|\le \dfrac{1}{2},$$ 
    for every color of points.
\end{block}
\begin{figure}[!ht]
\centering
\begin{subfigure}{0.49\textwidth}
\centering
\begin{tikzpicture}[thick, scale=0.4]
%%%%% Axis %%%%%
\draw[help lines, color=gray!30, dashed] (-4.9,-4.9) grid (4.9,4.9);
\draw[->,ultra thick] (-6,-5)--(5,-5) ;
\draw[->,ultra thick] (-5,-6)--(-5,5) ;
\draw[densely dashed,color=red,line width=1pt,opacity=1](-5,-2)--(0,-2)--(0,2)--(5,2);
\foreach \i in {-5,-4.5,...,-3,-2.5}{
\draw[densely dashed,color=blue,line width=0.5pt,opacity=1] (-5,\i)--(5,\i);
}
\foreach \i in {-2,-1.5,...,1,1.5}{
\draw[densely dashed,color=blue,line width=0.5pt,opacity=1] (0,\i)--(5,\i);
}
\node at (0,-5.5) {$i_1$};
\node at (-5.5,-2) {$j_1$};
\node at (-5.5,2) {$j$};
\end{tikzpicture}
\end{subfigure}
\begin{subfigure}{0.49\textwidth}
\centering
\begin{tikzpicture}[thick, scale=0.4]
%%%%% Axis %%%%%
\draw[help lines, color=gray!30, dashed] (-4.9,-4.9) grid (4.9,4.9);
\draw[->,ultra thick] (-6,-5)--(5,-5) ;
\draw[->,ultra thick] (-5,-6)--(-5,5) ;
\draw[densely dashed,color=red,line width=1pt,opacity=1](5,-2)--(0,-2)--(0,2)--(-5,2);
\foreach \i in {-5,-4.5,...,-3,-2.5}{
\draw[densely dashed,color=blue,line width=0.5pt,opacity=1] (-5,\i)--(5,\i);
}
\foreach \i in {-2,-1.5,...,1,1.5}{
\draw[densely dashed,color=blue,line width=0.5pt,opacity=1] (-5,\i)--(0,\i);
}
\node at (0,-5.5) {$i_1$};
\node at (-5.5,-2) {$j_1$};
\node at (-5.5,2) {$j$};

\end{tikzpicture}
\end{subfigure}
\caption{1 vertical and 2 horizontal segments.}
\end{figure}

\end{frame}
\subsection{5 Line Segments}
\begin{frame}[allowframebreaks]{Some Cases of 5 Line Segments}
\begin{figure}[!ht]
\centering
\begin{subfigure}{0.49\textwidth}
\centering
\begin{tikzpicture}[thick, scale=0.4]
%%%%% Axis %%%%%
\draw[help lines, color=gray!30, dashed] (-4.9,-4.9) grid (4.9,4.9);
\draw[->,ultra thick] (-6,-5)--(5,-5) ;
\draw[->,ultra thick] (-5,-6)--(-5,5) ;
\draw[densely dashed,color=red,line width=1pt,opacity=1] (2,-5)--(2,-2)--(-2,-2)--(-2,2)--(2,2)--(2,5);
\node at (-5.5,-2) {$i_1$};
\node at (-5.5,2) {$\color{red!80!black}i_2$};
\node at (-2,-5.5) {$j_1$};
\node at (2,-5.5) {$j$};
\foreach \i in {-5,-4.5,...,-3,-2.5}{
\draw[densely dashed,color=blue,line width=0.5pt,opacity=1] (\i,-5)--(\i,5);
}
\foreach \i in {-2,-1.5,...,1,1.5}{
\draw[densely dashed,color=blue,line width=0.5pt,opacity=1] (\i,-5)--(\i,-2);
\draw[densely dashed,color=blue,line width=0.5pt,opacity=1] (\i,2)--(\i,5);
}
\end{tikzpicture}
\end{subfigure}
\begin{subfigure}{0.49\textwidth}
\centering
\begin{tikzpicture}[thick, scale=0.4]
%%%%% Axis %%%%%
\draw[help lines, color=gray!30, dashed] (-4.9,-4.9) grid (4.9,4.9);
\draw[->,ultra thick] (-6,-5)--(5,-5) ;
\draw[->,ultra thick] (-5,-6)--(-5,5) ;
\draw[densely dashed,color=red,line width=1pt,opacity=1](-2,-5)--(-2,-2)--(2,-2)--(2,2)--(-2,2)--(-2,5);
\node at (-5.5,-2) {$i_1$};
\node at (-5.5,2) {$\color{red!80!black}i_2$};
\node at (-2,-5.5) {$j_1$};
\node at (2,-5.5) {$j$};
\foreach \i in {-5,-4.5,...,-3,-2.5}{
\draw[densely dashed,color=blue,line width=0.5pt,opacity=1] (\i,-5)--(\i,5);
}
\foreach \i in {-2,-1.5,...,1,1.5}{
\draw[densely dashed,color=blue,line width=0.5pt,opacity=1] (\i,-2)--(\i,2);
}
\end{tikzpicture}
\end{subfigure}
\caption{Some cases of 3 vertical and 2 horizontal segments.}
\end{figure}
\begin{block}{Convert into equation}
$$
    V=\sum_{i=0}^{i_1}a_{ij}+\sum_{i=i_1}^{i_2}a_{ij_1}+\sum_{i=i_2}^{n}a_{ij},$$
    \text{or}
    $$V=\sum_{i=0}^{i_1}a_{ij_1}+\sum_{i=i_1}^{i_2}a_{ij}+\sum_{i=i_2}^{n}a_{ij_1},
$$
where $0\le j_1\le j$ for each color of points.
\end{block}
\begin{block}{Verify a solution}
    This partition is a solution when $$\left|\dfrac{S_n}{2}-V\right|\le \dfrac{1}{2},$$ 
    for every color of points.
\end{block}
\begin{figure}[!ht]
\centering
\begin{subfigure}{0.49\textwidth}
\centering
\begin{tikzpicture}[thick, scale=0.4]
%%%%% Axis %%%%%
\draw[help lines, color=gray!30, dashed] (-4.9,-4.9) grid (4.9,4.9);
\draw[->,ultra thick] (-6,-5)--(5,-5) ;
\draw[->,ultra thick] (-5,-6)--(-5,5) ;
\draw[densely dashed,color=red,line width=1pt,opacity=1] (-5,2)--(-2,2)--(-2,-2)--(2,-2)--(2,2)--(5,2);
\node at (-5.5,-2) {$j_1$};
\node at (-5.5,2) {$j$};
\node at (-2,-5.5) {$i_1$};
\node at (2,-5.5) {$i_2$};
\foreach \i in {-5,-4.5,...,-3,-2.5}{
\draw[densely dashed,color=blue,line width=0.5pt,opacity=1] (-5,\i)--(5,\i);
}
\foreach \i in {-2,-1.5,...,1,1.5}{
\draw[densely dashed,color=blue,line width=0.5pt,opacity=1] (-5,\i)--(-2,\i);
\draw[densely dashed,color=blue,line width=0.5pt,opacity=1] (2,\i)--(5,\i);
}
\end{tikzpicture}
\end{subfigure}
\begin{subfigure}{0.49\textwidth}
\centering
\begin{tikzpicture}[thick, scale=0.4]
%%%%% Axis %%%%%
\draw[help lines, color=gray!30, dashed] (-4.9,-4.9) grid (4.9,4.9);
\draw[->,ultra thick] (-6,-5)--(5,-5) ;
\draw[->,ultra thick] (-5,-6)--(-5,5) ;
\draw[densely dashed,color=red,line width=1pt,opacity=1](-5,-2)--(-2,-2)--(-2,2)--(2,2)--(2,-2)--(5,-2);
\node at (-5.5,-2) {$j_1$};
\node at (-5.5,2) {$j$};
\node at (-2,-5.5) {$i_1$};
\node at (2,-5.5) {$i_2$};
\foreach \i in {-5,-4.5,...,-3,-2.5}{
\draw[densely dashed,color=blue,line width=0.5pt,opacity=1] (-5,\i)--(5,\i);
}
\foreach \i in {-2,-1.5,...,1,1.5}{
\draw[densely dashed,color=blue,line width=0.5pt,opacity=1] (-2,\i)--(2,\i);
}
\end{tikzpicture}
\end{subfigure}
\caption{Some cases of 2 vertical and 3 horizontal segments.}
\end{figure}

\end{frame}

\section{Result}
\begin{frame}{Counterexample of 3 Line Segments}
\begin{figure}[!ht]
    \centering
    \includegraphics[width=0.8\linewidth]{False 3colors100x100_pic78.png}
    \caption{Counterexample of equipartitions with 3 line segments for 3 color point sets.}
    \label{ex6}
\end{figure}
\end{frame}
\begin{frame}{5 Line Segments for 3-color Points}
\begin{figure}[!ht]
    \centering
    \includegraphics[width=0.8\linewidth]{True 3colors100x100_pic15}
    \caption{Equipartitions with 3 vertical and 2 horizontal segments for 3 color point sets when translated a value $x=j$.}
    \label{ex13}
\end{figure}
\end{frame}
\begin{frame}{Counterexample of some cases of 5 Line Segments}
\begin{figure}[!ht]
    \centering
    \includegraphics[width=0.8\linewidth]{False 4colors20x20_pic2094}
    \caption{Counterexample of equipartitions with some cases of 5 line segments for 4 color point sets.}
    \label{ex14}
\end{figure}
\end{frame}

\section{Discussion}
\begin{frame}[allowframebreaks]{General Case of 5 Line Segments}

% \end{frame}

% \begin{frame}{General Case of 5 Line Segments}

\begin{figure}[!ht]
\centering
\begin{subfigure}{0.24\textwidth}
\centering
\begin{tikzpicture}[thick, scale=0.21]
%%%%% Axis %%%%%
\draw[help lines, color=gray!30, dashed] (-4.9,-4.9) grid (4.9,4.9);
\draw[->,ultra thick] (-6,-5)--(5,-5) ;
\draw[->,ultra thick] (-5,-6)--(-5,5) ;
\draw[densely dashed,color=red,line width=1pt,opacity=1] (2,-5)--(2,-2)--(-2,-2)--(-2,2)--(0,2)--(0,5);
\foreach \i in {-5,-4.5,...,4.5,5}{
\draw[densely dashed,color=blue,line width=0.5pt,opacity=0.3] (\i,-5)--(\i,5);
}
\end{tikzpicture}
\end{subfigure}
\begin{subfigure}{0.24\textwidth}
\centering
\begin{tikzpicture}[thick, scale=0.21]
%%%%% Axis %%%%%
\draw[help lines, color=gray!30, dashed] (-4.9,-4.9) grid (4.9,4.9);
\draw[->,ultra thick] (-6,-5)--(5,-5) ;
\draw[->,ultra thick] (-5,-6)--(-5,5) ;
\draw[densely dashed,color=red,line width=1pt,opacity=1] (2,5)--(2,2)--(-2,2)--(-2,-2)--(0,-2)--(0,-5);
\foreach \i in {-5,-4.5,...,4.5,5}{
\draw[densely dashed,color=blue,line width=0.5pt,opacity=0.3] (\i,-5)--(\i,5);
}
\end{tikzpicture}
\end{subfigure}
\begin{subfigure}{0.24\textwidth}
\centering
\begin{tikzpicture}[thick, scale=0.21]
%%%%% Axis %%%%%
\draw[help lines, color=gray!30, dashed] (-4.9,-4.9) grid (4.9,4.9);
\draw[->,ultra thick] (-6,-5)--(5,-5) ;
\draw[->,ultra thick] (-5,-6)--(-5,5) ;
\draw[densely dashed,color=red,line width=1pt,opacity=1] (2,-5)--(2,-2)--(0,-2)--(0,2)--(-2,2)--(-2,5);
\foreach \i in {-5,-4.5,...,4.5,5}{
\draw[densely dashed,color=blue,line width=0.5pt,opacity=0.3] (\i,-5)--(\i,5);
}
\end{tikzpicture}
\end{subfigure}
\begin{subfigure}{0.24\textwidth}
\centering
\begin{tikzpicture}[thick, scale=0.21]
%%%%% Axis %%%%%
\draw[help lines, color=gray!30, dashed] (-4.9,-4.9) grid (4.9,4.9);
\draw[->,ultra thick] (-6,-5)--(5,-5) ;
\draw[->,ultra thick] (-5,-6)--(-5,5) ;
\draw[densely dashed,color=red,line width=1pt,opacity=1] (2,-5)--(2,2)--(0,2)--(0,-2)--(-2,-2)--(-2,5);
\foreach \i in {-5,-4.5,...,4.5,5}{
\draw[densely dashed,color=blue,line width=0.5pt,opacity=0.3] (\i,-5)--(\i,5);
}
\end{tikzpicture}
\end{subfigure}
\begin{subfigure}{0.24\textwidth}
\centering
\begin{tikzpicture}[thick, scale=0.21]
%%%%% Axis %%%%%
\draw[help lines, color=gray!30, dashed] (-4.9,-4.9) grid (4.9,4.9);
\draw[->,ultra thick] (-6,-5)--(5,-5) ;
\draw[->,ultra thick] (-5,-6)--(-5,5) ;
\draw[densely dashed,color=red,line width=1pt,opacity=1] (0,5)--(0,2)--(2,2)--(2,-2)--(-2,-2)--(-2,-5);
\foreach \i in {-5,-4.5,...,4.5,5}{
\draw[densely dashed,color=blue,line width=0.5pt,opacity=0.3] (\i,-5)--(\i,5);
}
\end{tikzpicture}
\end{subfigure}
\begin{subfigure}{0.24\textwidth}
\centering
\begin{tikzpicture}[thick, scale=0.21]
%%%%% Axis %%%%%
\draw[help lines, color=gray!30, dashed] (-4.9,-4.9) grid (4.9,4.9);
\draw[->,ultra thick] (-6,-5)--(5,-5) ;
\draw[->,ultra thick] (-5,-6)--(-5,5) ;
\draw[densely dashed,color=red,line width=1pt,opacity=1] (-2,5)--(-2,2)--(2,2)--(2,-2)--(0,-2)--(0,-5);
\foreach \i in {-5,-4.5,...,4.5,5}{
\draw[densely dashed,color=blue,line width=0.5pt,opacity=0.3] (\i,-5)--(\i,5);
}
\end{tikzpicture}
\end{subfigure}
\begin{subfigure}{0.24\textwidth}
\centering
\begin{tikzpicture}[thick, scale=0.21]
%%%%% Axis %%%%%
\draw[help lines, color=gray!30, dashed] (-4.9,-4.9) grid (4.9,4.9);
\draw[->,ultra thick] (-6,-5)--(5,-5) ;
\draw[->,ultra thick] (-5,-6)--(-5,5) ;
\draw[densely dashed,color=red,line width=1pt,opacity=1] (2,5)--(2,2)--(0,2)--(0,-2)--(-2,-2)--(-2,-5);
\foreach \i in {-5,-4.5,...,4.5,5}{
\draw[densely dashed,color=blue,line width=0.5pt,opacity=0.3] (\i,-5)--(\i,5);
}
\end{tikzpicture}
\end{subfigure}
\begin{subfigure}{0.24\textwidth}
\centering
\begin{tikzpicture}[thick, scale=0.21]
%%%%% Axis %%%%%
\draw[help lines, color=gray!30, dashed] (-4.9,-4.9) grid (4.9,4.9);
\draw[->,ultra thick] (-6,-5)--(5,-5) ;
\draw[->,ultra thick] (-5,-6)--(-5,5) ;
\draw[densely dashed,color=red,line width=1pt,opacity=1] (-2,-5)--(-2,2)--(0,2)--(0,-2)--(2,-2)--(2,5);
\foreach \i in {-5,-4.5,...,4.5,5}{
\draw[densely dashed,color=blue,line width=0.5pt,opacity=0.3] (\i,-5)--(\i,5);
}
\end{tikzpicture}
\end{subfigure}
\caption{General case of 3 vertical and 2 horizontal segments.}
\end{figure}
\begin{block}{Convert into equation}
$$
    V=\sum_{i=0}^{i_1}a_{ij_1}+\sum_{i=i_1}^{i_2}a_{ij_2}+\sum_{i=i_2}^{n}a_{ij_3},$$
    \text{where $i_1\le i_2$ for each color of points or}
    $$V=\sum_{i=0}^{i_1}a_{ij_1}-\sum_{i=i_1}^{i_2}a_{ij_2}+\sum_{i=i_2}^{n}a_{ij_3},
$$
where $i_1>i_2$ for each color of points.
\end{block}

\begin{figure}[!ht]
\centering
\begin{subfigure}{0.49\textwidth}
\centering
\begin{tikzpicture}[thick, scale=0.4]
%%%%% Axis %%%%%
\draw[help lines, color=gray!30, dashed] (-4.9,-4.9) grid (4.9,4.9);
\draw[->,ultra thick] (-6,-5)--(5,-5) ;
\draw[->,ultra thick] (-5,-6)--(-5,5) ;
\draw[densely dashed,color=red,line width=1pt,opacity=1] (2,-5)--(2,2)--(0,2)--(0,-2)--(-2,-2)--(-2,5);
\node at (-5.5,-2) {$i_2$};
\node at (-5.5,2) {$i_1$};
\node at (-2,-5.5) {$\color{red!80!black}j_3$};
\node at (0,-5.5) {$j_2$};
\node at (2,-5.5) {$j_1$};
\foreach \i in {-5,-4.5,...,-3,-2.5}{
\draw[densely dashed,color=blue,line width=0.5pt,opacity=1] (\i,-5)--(\i,5);
}
\foreach \i in {-2,-1.5,-1,-0.5,0}{
\draw[densely dashed,color=blue,line width=0.5pt,opacity=1] (\i,-5)--(\i,-2);
}
\foreach \i in {0.5,1,1.5}{
\draw[densely dashed,color=blue,line width=0.5pt,opacity=1] (\i,-5)--(\i,2);
}

\end{tikzpicture}
\end{subfigure}
\begin{subfigure}{0.49\textwidth}
\centering
\begin{tikzpicture}[thick, scale=0.4]
%%%%% Axis %%%%%
\draw[help lines, color=gray!30, dashed] (-4.9,-4.9) grid (4.9,4.9);
\draw[->,ultra thick] (-6,-5)--(5,-5) ;
\draw[->,ultra thick] (-5,-6)--(-5,5) ;
\draw[densely dashed,color=red,line width=1pt,opacity=1] (-2,-5)--(-2,2)--(0,2)--(0,-2)--(2,-2)--(2,5);
\node at (-5.5,-2) {$i_2$};
\node at (-5.5,2) {$i_1$};
\node at (-2,-5.5) {$j_1$};
\node at (0,-5.5) {$j_2$};
\node at (2,-5.5) {$\color{red!80!black}j_3$};
\foreach \i in {-5,-4.5,...,-3,-2.5}{
\draw[densely dashed,color=blue,line width=0.5pt,opacity=1] (\i,-5)--(\i,5);
}
\foreach \i in {-2,-1.5,-1,-0.5,0}{
\draw[densely dashed,color=blue,line width=0.5pt,opacity=1] (\i,2)--(\i,5);
}
\foreach \i in {0.5,1,1.5}{
\draw[densely dashed,color=blue,line width=0.5pt,opacity=1] (\i,-2)--(\i,5);
}
\end{tikzpicture}
\end{subfigure}
\caption{General case of 3 vertical and 2 horizontal segments with $i_1>i_2$.}
\end{figure}

\end{frame}
\begin{frame}
    \centering\Huge\textbf{\textcolor{msl!80!black}{Q \& A}}\emoji{thinking}\emoji{thinking}
    %\emoji{raising-hand-man}\emoji{person-raising-hand}\emoji{raising-hand-woman}% \emoji{thinking}\emoji{thinking}% \emoji{right-anger-bubble}\emoji{thought-balloon}
\end{frame}
\begin{frame}[allowframebreaks]{Application}
\begin{figure}
        \centering
        \includegraphics[width=1\linewidth]{download (31).png}
\end{figure}
\begin{figure}[ht]
    \centering
\begin{tikzpicture}[thick, scale=0.55]
\draw[help lines, color=gray!30, dashed] (-4.9,-4.9) grid (4.9,4.9);
\draw[->,ultra thick] (-5,0)--(5,0) ;
\draw[->,ultra thick] (0,-5)--(0,5) ;

\draw[fill=red,draw=black] (3,2.5) circle (3pt);
\draw[fill=red,draw=black] (-2.5,-3) circle (3pt);
\draw[fill=red,draw=black] (-1,3) circle (3pt);
\draw[fill=red,draw=black] (1,-3) circle (3pt);

\draw[fill=red,draw=black] (-3,1) circle (3pt);
\draw[fill=red,draw=black] (3,-1) circle (3pt);
\draw[fill=red,draw=black] (2.5,1) circle (3pt);
\draw[fill=red,draw=black] (-0.5,-2.5) circle (3pt);

\draw[fill=red,draw=black] (-0.5,0.5) circle (3pt);
\draw[fill=red,draw=black] (1,4) circle (3pt);
\draw[fill=red,draw=black] (-4,-1) circle (3pt);
\draw[fill=red,draw=black] (3,-2) circle (3pt);

\draw[fill=red,draw=black] (0.5,4.5) circle (3pt);
\draw[fill=red,draw=black] (-0.5,1) circle (3pt);
\draw[fill=red,draw=black] (-4.5,-1.5) circle (3pt);
\draw[fill=red,draw=black] (4,-4) circle (3pt);

\draw[fill=red,draw=black] (0.5,0.5) circle (3pt);
\draw[fill=red,draw=black] (-1,4) circle (3pt);
\draw[fill=red,draw=black] (-3,-2) circle (3pt);
\draw[fill=red,draw=black] (4,-1) circle (3pt);

\end{tikzpicture}
\end{figure}
\end{frame}
\begin{frame}[allowframebreaks]{References}
\def\newblock{}
\bibliographystyle{plain}
\bibliography{498Reference}
\end{frame}
\end{document}
